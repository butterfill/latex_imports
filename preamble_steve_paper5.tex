%!TEX TS-program = xelatex
%!TEX encoding = UTF-8 Unicode

\title{\ititle\\\isubtitle}
\author{\iauthor\\<{\iemail}>}

\usepackage[\papersize]{geometry} % see geometry.pdf
\geometry{twoside=false}
\geometry{headsep=2em} %keep running header away from text
\geometry{footskip=1cm} %keep page numbers away from text
\geometry{top=3cm} %increase to 3.5 if use header
\geometry{left=4cm} %increase to 3.5 if use header
\geometry{right=4cm} %increase to 3.5 if use header
\geometry{textheight=22cm}

%non-xelatex
%\usepackage[T1]{fontenc}
%\usepackage{tgpagella}

\usepackage{microtype}

%for underline
\usepackage[normalem]{ulem}

%get the font here:
% http://scripts.sil.org/CharisSILfont

\usepackage{fontspec,xunicode}
%nb do not explicitly use package xltxtra because this introduces bugs with footnote superscripting  -- perhaps because fontspec is supposed to include it anyway.
%UPDATE:  "You need to use the no-sscript option in xltxtra: \usepackage[no-sscript]{xltxtra}, this is explained in the documentation of xltxtra.  The issue is that Sabon does not contain true superscript glyphs for every character and the no-sscript option will instead use scaled regular glyphs, which is typographically inferior, but there is no other option available when using Sabon." --- http://groups.google.com/group/comp.text.tex/browse_thread/thread/19de95be2daacade
\defaultfontfeatures{Mapping=tex-text}
%\setromanfont[Mapping=tex-text]{Charis SIL} %i.e. palatino
%\setromanfont[Mapping=tex-text]{Adobe Jenson Pro}
%\setromanfont[Mapping=tex-text]{Dante MT Std}
% \setromanfont[Mapping=tex-text,Ligatures={Common}]{Hoefler Text} %comes with osx
%\setromanfont[Mapping=tex-text]{Bembo Book MT Pro}
% \setromanfont[Mapping=tex-text]{ETBembo}  %OTF version is Linux Libertine O but this stopped working on my machine!
\setromanfont[Mapping=tex-text,Ligatures={Common}]{Linux Libertine O} %comes with osx
\setsansfont[Mapping=tex-text]{Linux Biolinum O}
\setmonofont[Scale=MatchLowercase]{Andale Mono}




%handles references to labels (e.g. sections) nicely
\usepackage{varioref}

%hyperlinks and pdf metadata
%TODO avoid duplication of title & author
\usepackage{hyperref}
\hypersetup{pdfborder={0 0 0}}
\hypersetup{pdfauthor={\iauthor}}
\hypersetup{pdftitle={\ititle\isubtitle}}

%handles references to labels (e.g. sections) nicely
\usepackage{cleveref}
\crefname{figure}{figure}{figures}
\crefname{chapter}{Chapter}{Chapters}

%line spacing
\usepackage{setspace}
%\onehalfspacing
%\doublespacing
%\singlespacing


\usepackage{natbib}
%\usepackage[longnamesfirst]{natbib}
\setcitestyle{aysep={}}  %philosophy style: no comma between author & year

%% for urls in bibliography
%% http://www.kronto.org/thesis/tips/url-formatting.html
\usepackage{url}
%% Define a new 'leo' style for the package that will use a smaller font.
\makeatletter
\def\url@leostyle{%
  \@ifundefined{selectfont}{\def\UrlFont{\sf}}{\def\UrlFont{\small\ttfamily}}}
\makeatother
%% Now actually use the newly defined style.
\urlstyle{leo}


%enable notes in right margin, defaults to ugly orange boxes TODO fix
%\usepackage[textwidth=5cm]{todonotes}

%for comments
\usepackage{verbatim}

%footnotes
\usepackage[hang,bottom,stable]{footmisc}
% no space between multiple paragraphs  in footnote
\renewcommand{\hangfootparskip}{0em}
% multiple paragraphs  in footnote are indented by 1em
\renewcommand{\hangfootparindent}{1em}
\setlength{\footnotemargin}{1em}
\setlength{\footnotesep}{1em}
\footnotesep 2em

%tables
\usepackage{booktabs}
\usepackage{ctable}
\usepackage{array} %allows m columns in tables (paragraph, vertically centered)
\usepackage{tabu}

%section headings
\usepackage[rm]{titlesec} %sf for sans, rm for roman
%\titlespacing*{\section}{0pt}{*3}{*0.5} %reduce vertical space after header
%large headings:
%\titleformat{\section}{\LARGE\sffamily}{\thesection.}{1em}{}
\titlelabel{\thetitle.\quad} %make dot after section number

%captions
\usepackage[font={small,rm}, margin=0.75cm]{caption}

%lists
\usepackage{enumitem}
\newenvironment{idescription}
{
	% begin code
	\begin{description}[
		labelindent=1.5\parindent,
		leftmargin=2.5\parindent
	]
}
{
	%end code
	\end{description}
}


%title
\usepackage{titling}
\pretitle{
	\begin{center}
	%\sffamily %for sans title
	\LARGE % \Huge
}
\posttitle{
	\par
	\end{center}
	\vskip 0.5em
}
\preauthor{
	\begin{center}
	\normalsize
	\lineskip 0.5em
	\begin{tabular}[t]{c}
}
\postauthor{
	\end{tabular}
	\par
	\end{center}
}
\predate{
	\begin{center}
	\normalsize
}
\postdate{
	\par
	\end{center}
}
